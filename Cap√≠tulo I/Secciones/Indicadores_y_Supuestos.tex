\section{Indicadores}
\begin{itemize}
	\item(Sevilla y Córdoba, 2014) El cambio en las metodologías de aula viene de la mano, en muchos casos, de Internet y de las herramientas de la Web 2.0. Por otra parte, el desarrollo de una perspectiva de corte constructivista apoyado en el trabajo en grupo, suponen que la formación de los estudiantes puede ser alimentada a través de este tipo de recursos, dado que poten- cia, entre otros aspectos la socialización, la búsqueda de información, el logro de una meta común, etc.
	Un informe de Cabero Almenara y Marín Díaz (2014) sugiere que el acceso a recursos educativos en línea puede mejorar el rendimiento académico y la satisfacción de los estudiantes, lo que a su vez puede disminuir la deserción escolar.
	\item Fundación Telefónica (2016) .Hoy los jóvenes acceden a la información a través de la Web, construyen sus comunidades utilizando los dispositivos móviles, se comunican en las redes y se divierten jugando en entornos virtuales. Las escuelas apenas han incorporado en sus protocolos esta realidad (cuando no directamente la rechazan y estigmatizan); los educadores, por su parte, en su gran mayoría no están formados para este nuevo entorno. 
	
	Un informe de la Fundación Telefónica (2016) sugiere que el acceso a información educativa en línea puede ayudar a los padres a involucrarse más en la educación de sus hijos, ya que les permite estar más informados sobre el progreso académico y las actividades escolares de sus hijos.
	
\end{itemize}
\section{Supuestos}
\begin{itemize}
	\item El involucramiento de la familia en la educación infantil se ha relacionado con un mejor rendimiento académico en los niños. Un estudio realizado por Fan y Chen (2001) encontró que el apoyo y la participación de los padres en la educación de sus hijos están asociados con un mayor rendimiento académico, incluyendo mejores calificaciones y habilidades de lectura
	\item Un estudio de McWayne, Hampton, Fantuzzo, Cohen y Sekino (2004) encontró que el involucramiento familiar en la educación temprana se relaciona con un mejor ajuste socioemocional en los niños, incluyendo habilidades sociales, autocontrol y comportamiento prosocial.
	La participación activa de la familia en la educación de los niños también puede mejorar su desarrollo socioemocional.
	\item La participación de la familia en la educación infantil también puede influir en las actitudes y la motivación de los niños hacia el aprendizaje. Un estudio de Jeynes (2005) encontró que el involucramiento de los padres en la educación de sus hijos está relacionado con una mayor motivación y actitudes más positivas hacia la escuela y el aprendizaje. (Jeynes, W. H. 2005).
\end{itemize}