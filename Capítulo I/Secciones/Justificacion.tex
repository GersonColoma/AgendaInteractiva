\section{Justificación}
\hspace{2cm}Las prácticas que surgen por la necesidad de adaptarse de las personas hacen que también
se transformen las costumbres y los dogmas establecidos por las sociedades. Las costumbres
cambian con el paso del tiempo e incluir éstos cambios a la rutina también, tarde o temprano se
convierten en una necesidad; la cual termina siendo urgente cuando el costumbrismo y la
negación al cambio desembocan en la perturbación del desarrollo natural y el florecimiento
adecuado de la vida, el intelecto y las capacidades completas de cada niño y niña en los distintos
núcleos familiares.
Las necesidades y la incertidumbre del día a día, las dificultades, la mala interpretación
situacional; también la negativa a involucrarse realmente con la vida que nos rodea y con los
roles en las distintas áreas de experiencia del día a día, hacen que las personas a veces pierdan la
noción de que todos vivimos en el mismo espació es decir que compartimos un mundo, lo que se
traduce en ideas erradas, poco inclusivas, en egoísmo y confusión; pero también en el
desentendimiento de las obligaciones de cada parte en éste caos que buscamos ordenar, llamado
sociedad.


\vspace{1cm}Una situación problemática en el involucramiento de padres y maestros en la educación
infantil podría ser la falta de comunicación efectiva entre las partes. Cuando los padres y maestros no se comunican regularmente y no comparten información relevante sobre el progreso del niño o niña, se puede crear una brecha en la comprensión de las necesidades del estudiante y
de sus fortalezas y debilidades. Además, la falta de comunicación puede obstaculizar la
capacidad de padres y maestros para colaborar en la resolución de problemas o para diseñar un
plan educativo que atienda específicamente las necesidades individuales del estudiante. Esta falta
de coordinación y colaboración puede ser perjudicial para el niño o niña y dificultar su progreso
académico y social. Por lo tanto, es importante que los padres y maestros se comuniquen
regularmente y desarrollen estrategias efectivas para trabajar juntos en la educación del infante.

El involucramiento de padres y maestros es esencial para el éxito académico y personal
de los niños y niñas. Cuando padres y maestros trabajan juntos, los niños experimentan una
educación más completa y significativa.
Los padres están en una posición única para apoyar el aprendizaje de sus hijos. Pueden
ayudar con la tarea, asistir a las conferencias de padres y maestros, y brindar apoyo emocional y
motivacional para sus hijos. La implicación de los padres también puede ayudar a crear una
cultura de alta expectativa y colaboración en la escuela.
Por su parte, los maestros pueden colaborar con los padres para crear experiencias de
aprendizaje que sean relevantes y significativas para los niños. Los maestros pueden utilizar la
información que los padres brindan para entender las necesidades de los niños y adaptar el
aprendizaje en consecuencia.
En la actualidad, la pandemia ha aumentado la importancia del involucramiento de padres
y maestros, ya que ha creado nuevos desafíos para el aprendizaje en el hogar. La colaboración
entre los padres y los maestros es fundamental para asegurarse de que los niños sigan siendo
motivados e involucrados en su educación.
En resumen, la colaboración entre padres y maestros es vital para el éxito de los niños y
niñas. Su implicación puede aumentar la motivación y el aprendizaje de los niños, y ayudarlos a
desarrollar habilidades para el éxito a largo plazo. Esto es particularmente importante en tiempos
de crisis, como la pandemia actual.

La presente investigación surge de la necesidad de estudiar el des-atendimiento de parte
del núcleo familiar hacia los niños en pre-primaria y primaria en general privada, con el
propósito de identificar la cantidad de casos ocurridos, tal como lo manifieste el rendimiento en
sí del estudiante y las impresiones de los maestros como tal, así como las estrategias de
prevención implementadas por la institución y los educadores ante este tipo de conductas
antisociales.

Una agenda interactiva, tanto para maestros, así como para los padres e hijos que
proporcione información de utilidad para la comunidad educativa con tal de mejorar el
conocimiento sobre el problema en la institución y en ese lapso de corte de comunicación que
surge luego de la coyuntura de ideales y barreras que presenta la interacción padre-maestro,
puesta al día constantemente de los acontecimientos, retroalimentación y toma en cuenta de las
impresiones de los estudiantes es vital plasmarla en la agenda para abarcar la mayor cantidad de
variables y tenerlas presentes a la hora de la reportería.
Lo que se busca con esta agenda interactiva, es lograr que los padres y maestros puedan
saber qué hacer ante la situaciones de perder el hilo del desempeño del niño o niña, que no
simplemente lo dejen por su cuenta, sino que lo apoyen y juntos busquen más ayuda antes de que
se establezca un gran problema que está sucediendo hoy en día, no solo en los niños sino también
en las personas adultas que es la falta de inclusión y motivación para pertenecera a una sociedad
sana, cooperativa, inclusiva y activa.

Lo que se pretende conseguir es que los padres, maestros o incluso los encargados puedan
atender las necesidades que pueda tener el estudiante, no solo materialmente, sino también
psicológicamente, que lo sepan ayudar, motivar e instruir a lo largo de su camino académico
siempre manteniendo los temas de estudio centrados en las necesidades del dìa y en sus
acontecimientos, evitando que surjan desvíos o mal interpretaciones contextuales entre padres y
maestros asì como una retroalimentaciòn completa entre èstos mismos, dada tanto por los adultos
como por los pequeños de la historia, los niños y niñas estudiantes que al final son los màs
afectados en èsta dicotomìa y los que deben ser más escuchados y tomados en cuenta en
cualquier decisión.