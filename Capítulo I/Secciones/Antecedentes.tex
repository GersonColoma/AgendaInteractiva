\section{Antecedentes}

\hspace{2cm}La educación en línea y las tecnologías de la información pueden proporcionar flexibilidad a los estudiantes, lo que puede ser especialmente útil para aquellos que enfrentan
desafíos personales o familiares. Durante la pandemia por Coronavirus en 2020 se vivió un cambio relevante en lo relativo
al uso de las TIC en la educación. El sistema educativo mexicano, como el de otros países, se
tuvo que adaptar a la situación de emergencia para procurar la continuidad académica. Si bien
este hecho no implicó cambios a nivel tecnológico, si representó un incremento en la intensidad
de uso de las TIC en el contexto educativo específicamente en plataformas virtuales, sistemas de
videoconferencia y grupos de colaboración mediante redes sociales o mensajeros instantáneos
para la gestión de la educación. (Casillas y Martinel, 2020).

La comunicación y colaboración en línea pueden mejorar el compromiso y la satisfacción
de los estudiantes, lo que puede reducir la deserción escolar.
Hablar de “familias” y de “escuelas” podría sugerir una imagen —engañosa— de dos
“bloques” enfrentados, homogéneos en el interior de cada uno pero sustancialmente diferentes
uno del otro. Identificamos este tipo de imágenes en distintas expresiones del sentido común
—por ejemplo, en la forma en que muchas veces se plantea el tema en los medios de
comunicación masiva—, así como en distintas producciones académicas (tales como Benegas y
Verstraete, 2005; Gómez Schettini, 2007); o en diversos trabajos vinculados con el concepto de
“educabilidad” (Navarro, 2003; López y Tedesco, 2002; entre otros). Como señalan Neufeld y
Thisted (2004), dichos textos, al abordar el concepto de educabilidad, tienden a contemplar a las
familias en términos individualizados, culpabilizándolas por el llamado fracaso escolar, “pero se
dejan entre paréntesis las prácticas pedagógicas y los contextos escolares en que suceden éxitos y
fracasos” (Neufeld y Thisted, 2004, p. 83). (Cerletti, 2009).

Una adecuada implementación de la tecnología en el entorno escolar y familiar, la cual
permita tener un medio de comunicación inmediato, controlado y de calidad; es la base para
iniciar un camino de inclusión de todas las partes interesadas en la educación de los niños y
niñas.
El uso de las nuevas tecnologías es la base para la construcción de entornos de e-learning.
Pese a que en los últimos años se están generando numerosas iniciativas para la incorporación de
estos entornos al aula, este tipo de herramientas se encuentra poco implantado. La enseñanza
sigue siendo de profesor a alumno, sin considerar el llamado \&quot;aprendizaje colectivo\&quot; en el que
las redes sociales participan aportando sus conocimientos a los restantes usuarios. En ese modelo
educativo el acceso a la información en cada intervención genera nuevos puntos de vista que
enriquecen el conocimiento. El procedimiento está centrado en el propio alumno y permite,
gracias a la difusión multimedia, el avance intelectual por caminos variados promoviendo la
investigación y la exploración, así como la formación de opiniones. Loreto Corredoira y
Alfonso Española, 2012  
