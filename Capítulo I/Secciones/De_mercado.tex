\section{De Mercado}
\hspace{1.47cm}Existen varias aplicaciones para instituciones educativas, pero la mayoría ofrecen sistemas de gestión de tareas, calificaciones y contenido didáctico, sin embargo no están enfocadas en el seguimiento personalizado y menos en la interacción entre las partes que la ocupan, sacar de las redes sociales las interacciones educativas y profesionales y brindar una alternativa a éstas para los interesados; la gestión del tiempo de interacción, horarios de atención y asistencia para padres y alumnos, así como la retroalimentación y calificación entre sí son aspectos que no suelen tomar en cuenta otras alternativas que solo se basan en notas, contabilidad y números.
\subsection{Tecnológico}
\hspace{1.47cm}La implementación de sistemas automatizados, de interacción o de control; con las facilidades que se poseen en gran medida hoy en dìa son de carácter urgente si se tiene el interés de mantener las instituciones a la vanguardia
\subsection{Soporte}
\hspace{1.47cm}La retroalimentación y adaptabilidad son posibles hoy en día con más facilidad y rapidez por lo que una aplicación en constante evolución y por qué no, escalable, igualmente es lo ideal en la mayoría de los casos.
\subsection{Empresariales}
\hspace{1.47cm}El colegio manifiesta la necesidad de mantener y mantenerse informado de los acontecimientos que ocurren con los estudiantes, actualizar su información rápidamente y resolver dudas de manera bidireccional en determinados momentos incluso instantáneamente.