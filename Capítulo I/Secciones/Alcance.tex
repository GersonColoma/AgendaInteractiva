\section{Alcance}
\subsection{Geográfico}
\hspace{1.47cm}El estudio se desarrollará en un colegio privado ubicado en Buena Vista Chimaltenango.
\subsection{Tecnológicos}
\hspace{1.47cm}Se utilizará un editor de texto avanzado, la consola de comandos y la consola de FIREBASE en forma de backend como servicio el cuál desde sus servicios gratuitos ya ofrece certificados de seguridad como el HTTPS y SSL, además de varias opciones más así como reglas de seguridad específicas para distintos tipos de datos. Los lenguajes a utilizar principalmente serán JAVASCRIPT, CSS, HTML5 como lenguaje de alto nivel, diseño y maquetado; como base de datos se utilizarán las herramientas proporcionadas por FIREBASE así como su correspondiente servicio de alojamiento para soportar la aplicación, un sistema de control de usuarios para mantener controlados los contenidos a los que se tiene acceso y los datos de los usuarios.
\subsection{Personales}
\hspace{1.47cm}Colaboradores de la institución que tengan la autorización para acceder al sistema.
\subsection{Temporal}
\hspace{1.47cm}De febrero a noviembre del año 2023.
\subsection{Temático}
\hspace{1.47cm}Se desarrollará una aplicación web progresiva por su fácil adaptabilidad en distintos dispositivos y la facilidad de implementarla en distintos ámbitos. Con ésto se espera agilizar la interacción entre los padres, maestros y estudiantes de la institución para facilitar la resolución de dudas de ambas partes la adecuada toma encuenta situacional más importante de los niños y un flujo constante pero controlado de interacción de padre y maestros.