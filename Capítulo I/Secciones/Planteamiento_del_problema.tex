\section{Planteamiento del Problema}

\hspace{2cm}El poco involucramiento de los padres y maestros en la educación de los niños es una
preocupación cada vez más común en la sociedad actual. La falta de participación de los padres y
maestros en la formación de los niños puede tener consecuencias negativas en el aprendizaje y
desarrollo de los niños.

Una directora de grupo, especialista en humanidades, con una orientadora hacen lo posible
con la ayuda de Dios. Pero se necesita más que eso, estos muchachos (as) necesitan especialistas
en muchas áreas (psiquiatría, terapia familiar, psicología, fonoaudiología, trabajo social, hasta un
abo- gado y quizás otros), pero sobre todo queridos padres estos muchachos necesitan tener una
familia, necesitan tener afecto y quien los escuche. (Ardila Serrano, 1998).

\vspace{1cm}Una de las razones por las que los padres y maestros se mantienen al margen puede ser la
falta de tiempo y energía para abordar el tema de la educación de los niños. Los padres pueden
estar ocupados en el trabajo, en la casa o en otras actividades, y los maestros pueden tener una
carga excesiva de trabajo. Además, muchos padres pueden sentirse incómodos o inseguros al
intervenir en el ámbito educativo, ya que no se sienten capaces de brindar la orientación
necesaria a sus hijos.

El problema de la educabilidad apunta a la calidad de un arreglo institucional entre Estado,
familia y sociedad civil, y el fortalecimiento o deterioro de las condiciones de educabilidad
resulta de cambios en la relación entre estas esferas, desajustes entre lo que el niño trae y lo que
la escuela exige” (Lopez y Tedezco, 2002).

El poco involucramiento de los padres y maestros en la educación de los niños también
puede deberse a una falta de confianza en el sistema educativo. Muchos padres y maestros
pueden sentir que no tienen control sobre la educación de los niños, ya sea porque no confían en
las políticas educativas o porque no creen que el sistema satisfaga las necesidades de los niños.
En este sentido, es fundamental que las políticas educativas se diseñen en colaboración con los
padres y maestros, para así poder construir un sistema educativo en el que todos tengan voz y
voto, dando pie a una comunicación activa de las tres partes, alumno, padre y maestro; con el
objetivo de agilizar la toma de decisiones, evitando las actitudes negativas de parte de cualquiera
de los tres antes mencionados, la implementación de la agenda inteligente proporciona
participación activa de las partes, toma en cuenta los enfoques y mantiene la interacción en un
entorno controlado fuera de las redes sociales además de un nuevo enfoque y datos que puedan
generar un punto de vista externo e imparcial que ayudará a identificar el problema en algún
determinado caso que lo requiera.

La tesis central del documento es que en vastos sectores de la sociedad que habían logrado
incorporar a sus niños y adolescentes al sistema educativo, las condiciones de educabilidad se
están deteriorando porque las familias ya no pueden asumir el compromiso de garantizar su
preparación para las exigencias de la escuela, y porque la escuela no modifica su oferta de un
modo que permita compensar estos déficits en las capacidades de las familias” (Lopez y
Tedezco, 2002)

Por último, el poco involucramiento de los padres y maestros en la educación de los niños
puede derivar en la desmotivación y el desinterés de los niños por el aprendizaje. Los niños
necesitan sentir que sus padres y maestros están comprometidos en su formación, para así estar
motivados y dispuestos a aprender. Si los padres y maestros no están interesados en el proceso de
aprendizaje de los pequeños, los niños pueden perder la motivación y el interés por aprender.
En conclusión, el involucramiento de los padres y maestros en la educación de los niños es
fundamental para un adecuado desarrollo y aprendizaje de los mismos. Es importante que los
padres y maestros asuman un rol activo en la educación de los niños, ya que esto ayudará a
generar la confianza y motivación necesarias para garantizar un aprendizaje significativo.